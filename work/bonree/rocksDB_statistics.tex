% Created 2020-03-12 Thu 10:46
% Intended LaTeX compiler: pdflatex
\documentclass[11pt]{article}
\usepackage[utf8]{inputenc}
\usepackage[T1]{fontenc}
\usepackage{graphicx}
\usepackage{grffile}
\usepackage{longtable}
\usepackage{wrapfig}
\usepackage{rotating}
\usepackage[normalem]{ulem}
\usepackage{amsmath}
\usepackage{textcomp}
\usepackage{amssymb}
\usepackage{capt-of}
\usepackage{hyperref}
\author{王超}
\date{\today}
\title{rocksDB目录树测试}
\hypersetup{
 pdfauthor={王超},
 pdftitle={rocksDB目录树测试},
 pdfkeywords={},
 pdfsubject={},
 pdfcreator={Emacs 26.3 (Org mode 9.1.14)}, 
 pdflang={English}}
\begin{document}

\maketitle
\tableofcontents

\section{最好情况,一直命中缓存}
\label{sec:org7bd7530}
\begin{center}
\begin{tabular}{lrrrrrrrr}
目录层数 & 1 & 5 & 10 & 20 & 50 & 100 & 150 & 200\\
read time(micros) & 1612 & 1720 & 1860 & 1705 & 1698 & 1722 & 2145 & 2467\\
\end{tabular}
\end{center}

\section{最坏情况}
\label{sec:org07f3995}
一直不能命中缓存,但有读的时候把下一次要读的正好缓存上来了
\begin{center}
\begin{tabular}{lrrrrrrrr}
目录层数 & 1 & 5 & 10 & 20 & 50 & 100 & 150 & 200\\
read time(micros) & 7306 & 7565 & 7782 & 7564 & 7521 & 7990 & 8761 & 7846\\
\end{tabular}
\end{center}

\section{最最坏情况}
\label{sec:orgdb771bd}
排除顺序扫的可能
\begin{center}
\begin{tabular}{lrrrrrrrr}
目录层数 & 1 & 5 & 10 & 20 & 50 & 100 & 150 & 200\\
read time(micros) & 7050 & 7162 & 9662 & 10296 & 12500 & 15912 & 19088 & 30905\\
\end{tabular}
\end{center}
\end{document}
